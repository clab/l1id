\documentclass[landscape,final]{baposter}

\usepackage{times}
\usepackage{calc}
\usepackage{graphicx}
\usepackage{amsmath}
\usepackage{amssymb}
\usepackage{relsize}
\usepackage{multirow}
\usepackage{url}
\usepackage{tipa}
%\usepackage{biblatex}

\usepackage{multicol}
\usepackage{natbib}
\newcommand{\namecite}{\citet}
\renewcommand{\cite}{\citep}
\bibpunct[, ]{[}{]}{;}{n}{,}{,}

\usepackage{pgfbaselayers}
\pgfdeclarelayer{background}
\pgfdeclarelayer{foreground}
\pgfsetlayers{background,main,foreground}

\usepackage{helvet}
\usepackage{palatino}
\usepackage{listings}
\newcommand{\captionfont}{\footnotesize}
\newcommand{\hl}[1]{\color{red}{#1}}
%\newcommand{\textgl}[1]{``#1''}
%\newcommand{\word}[2]{\textnl{#1} \textgl{#2}}
%\newcommand{\gloss}[3]{\textnl{#1} (\textnl{#2}) \textgl{#3}}

%\newcommand{\upspc}{\vspace{-1.9em}}

\newcommand{\feat}[1]{\textsmaller[.5]{\textsf{#1}}} % code for a feature group
%\newcommand{\featstr}[1]{\textsmaller[.5]{\begin{lstlisting}#1\end{lstlisting}}} % full feature name
\newcommand{\textnl}{\textit}

\selectcolormodel{cmyk}

\graphicspath{{../}}

%%%%%%%%%%%%%%%%%%%%%%%%%%%%%%%%%%%%%%%%%%%%%%%%%%%%%%%%%%%%%%%%%%%%%%%%%%%%%%%%
% Multicol Settings
%%%%%%%%%%%%%%%%%%%%%%%%%%%%%%%%%%%%%%%%%%%%%%%%%%%%%%%%%%%%%%%%%%%%%%%%%%%%%%%%
\setlength{\columnsep}{0.7em}
\setlength{\columnseprule}{0mm}


%%%%%%%%%%%%%%%%%%%%%%%%%%%%%%%%%%%%%%%%%%%%%%%%%%%%%%%%%%%%%%%%%%%%%%%%%%%%%%%%
% Save space in lists. Use this after the opening of the list
%%%%%%%%%%%%%%%%%%%%%%%%%%%%%%%%%%%%%%%%%%%%%%%%%%%%%%%%%%%%%%%%%%%%%%%%%%%%%%%%
\newcommand{\compresslist}{%
\setlength{\itemsep}{1pt}%
\setlength{\parskip}{0pt}%
\setlength{\parsep}{0pt}%
}


%%%%%%%%%%%%%%%%%%%%%%%%%%%%%%%%%%%%%%%%%%%%%%%%%%%%%%%%%%%%%%%%%%%%%%%%%%%%%%
%%% Begin of Document
%%%%%%%%%%%%%%%%%%%%%%%%%%%%%%%%%%%%%%%%%%%%%%%%%%%%%%%%%%%%%%%%%%%%%%%%%%%%%%

\begin{document}

%%%%%%%%%%%%%%%%%%%%%%%%%%%%%%%%%%%%%%%%%%%%%%%%%%%%%%%%%%%%%%%%%%%%%%%%%%%%%%
%%% Here starts the poster
%%%---------------------------------------------------------------------------
%%% Format it to your taste with the options
%%%%%%%%%%%%%%%%%%%%%%%%%%%%%%%%%%%%%%%%%%%%%%%%%%%%%%%%%%%%%%%%%%%%%%%%%%%%%%
\typeout{Poster Starts}
\background{
  \begin{tikzpicture}[remember picture,overlay]%
    \draw (current page.north west)+(-2em,-0em) node[anchor=north west] {\hspace{-2em}\includegraphics[height=1.1\textheight]{silhouettes_background}};
  \end{tikzpicture}%
}
\definecolor{silver}{cmyk}{0,0,0,0.3}
\definecolor{yellow}{cmyk}{0,0,0.9,0.0}
\definecolor{reddishyellow}{cmyk}{0,0.22,1.0,0.0}
\definecolor{black}{cmyk}{0,0,0.0,1.0}
\definecolor{darkYellow}{cmyk}{0,0,1.0,0.5}
\definecolor{darkSilver}{cmyk}{0,0,0,0.1}

\definecolor{lightyellow}{cmyk}{0,0,0.3,0.0}
\definecolor{lighteryellow}{cmyk}{0,0,0.1,0.0}
\definecolor{lighteryellow}{cmyk}{0,0,0.1,0.0}
\definecolor{lightestyellow}{cmyk}{0,0,0.05,0.0}
% My colors
\definecolor{lightmagenta}{cmyk}{0,0.3,0.0,0.0}
\definecolor{lightermagenta}{cmyk}{0,0.1,0,0.0}
\definecolor{lightestmagenta}{cmyk}{0,0.05,0.0,0.0}
\definecolor{cyan}{cmyk}{0.9,0.,0.0,0.0}
\definecolor{lightcyan}{cmyk}{0.3,0.,0.0,0.0}
\definecolor{lightercyan}{cmyk}{0.1,0,0,0.0}
\definecolor{lightestcyan}{cmyk}{0.05,0.0,0.0,0.0}
\definecolor{blue}{cmyk}{0.5,0.2,0.0,0.0}
\definecolor{lightblue}{cmyk}{0.2,0.12,0.0,0.0}
\definecolor{lighterblue}{cmyk}{0.1,0.05,0,0.0}
\definecolor{lightestblue}{cmyk}{0.01,0.005,0.0,0.0}

\definecolor{darkpurple}{cmyk}{1,1,0.0,0.0}
\definecolor{lightpurple}{cmyk}{0.1,0.4,0.0,0.0}

\begin{poster}{
  % Show grid to help with alignment
  grid=no,
  % Column spacing
  colspacing=0.8em,
  % Color style
  bgColorOne=lightestcyan, %lighteryellow,
  bgColorTwo=lighterblue, %lightestyellow,
  borderColor=darkpurple, %reddishyellow,
  headerColorOne=blue, %yellow,
  headerColorTwo=darkpurple, %reddishyellow,
  headerFontColor=black,
  boxColorOne=lighterblue, %lightyellow,
  boxColorTwo=lighterblue, %lighteryellow,
  % Format of textbox
  textborder=rounded, %roundedleft,
  % Format of text header
  eyecatcher=yes,
  headerborder=closed,
  headerheight=0.1\textheight,
  headershape=rounded, %roundedright,
  headershade=plain,
  headerfont=\Large\textsf, %Sans Serif
  boxshade=plain,
%  background=shade-tb,
  background=plain,
  linewidth=2pt
  }
  % Eye Catcher
  {%\begin{minipage}{7em}
\includegraphics[width=9em]{img/cmu_logo}\\
%\end{minipage}
} % No eye catcher for this poster. If an eye catcher is present, the title is centered between eye-catcher and logo.
  % Title
  {\bf %Sans Serif
  %\bf% Serif
  Identifying the native language of non-native English writers
  }
    % Authors
  {\large\sl %Sans Serif
  % Serif
  Chris Dyer, Manaal Faruqui, Nathan Schneider, Yulia Tsvetkov, Language Technologies Institute,
  Carnegie Mellon University, Pittsburgh, PA
  \large
  \\Noam Ordan, Naama Twitto, Shuly Wintner,
  Department of Computer Science,University of Haifa, Haifa, Israel
  }
  % University logo clip=true, trim=l b r t,
  {  \includegraphics[height=5.1em]{img/haifa_univ_logo} %logo1
  }
 % {  \includegraphics[height=5.1em]{img/logo1}
  %}

  \tikzstyle{light shaded}=[top color=baposterBGtwo!30!white,bottom color=baposterBGone!30!white,shading=axis,shading angle=30]

  % Width of left inset image
   %  \newlength{\leftimgwidth}
    % \setlength{\leftimgwidth}{0.78em+8.0em}

%%%%%%%%%%%%%%%%%%%%%%%%%%%%%%%%%%%%%%%%%%%%%%%%%%%%%%%%%%%%%%%%%%%%%%%%%%%%%%
%%% Now define the boxes that make up the poster
%%%---------------------------------------------------------------------------
%%% Each box has a name and can be placed absolutely or relatively.
%%% The only inconvenience is that you can only specify a relative position 
%%% towards an already declared box. So if you have a box attached to the 
%%% bottom, one to the top and a third one which should be in between, you 
%%% have to specify the top and bottom boxes before you specify the middle 
%%% box.
%%%%%%%%%%%%%%%%%%%%%%%%%%%%%%%%%%%%%%%%%%%%%%%%%%%%%%%%%%%%%%%%%%%%%%%%%%%%%%
    %
    % A coloured circle useful as a bullet with an adjustably strong filling
    \newcommand{\colouredcircle}[1]{%
      \tikz{\useasboundingbox (-0.2em,-0.32em) rectangle(0.2em,0.32em); \draw[draw=black,fill=bgColorOne!80!black!#1!white,line width=0.03em] (0,0) circle(0.18em);}}

    \newcommand{\numbox}[1]{%
      \tikz{\useasboundingbox (-0.2em,-0.32em) rectangle(0.2em,0.32em); \draw[draw=black,fill=headerColorOne!80!black!80!white,line width=0.03em] (0,0) circle(0.5em); \path (0,0) node {\smaller #1};}\hspace{0.2em}}

%%%%%%%%%%%%%%%%%%%%%%%%%%%%%%%%%%%%%%Abstract%%%%%%%%%%%%%%%%%%%%%%%%%%%%%%%%%%%%%%%
\headerbox{Abstract}{name=abstract,column=0,row=0,span=1.15}{
{\bf We show that it is possible to identify, with high accuracy, the native language of English test takers from the content of the essays they write.} 
Our method uses standard text classification techniques based on multiclass logistic regression, combining individually weak indicators to predict the most probable native language from a set of 11 possibilities: Arabic, Chinese, French, German, Hindi, Italian, Japanese, Korean, Spanish, Telugu, and Turkish.
}

%%%%%%%%%%%%%%%%%%%%%%%%%%%%%%%
\headerbox{Classification Model}{name=model,column=0,span=1.15,below=abstract}{
We use a logistic regression classifier implemented by \texttt{creg} trained to maximize the log-likelihood of the training data, penalized by a combined $\ell_2$ and entropic regularizer.
\begin{align*}
\hat{\boldsymbol{\lambda}} = \arg \min_{\boldsymbol{\lambda}} \alpha \overbrace{\sum_j \lambda_j^2}^{\textrm{$\ell_2$ reg.}} - \sum_{\{(\textbf{x}_i,y_i )\}_{i=1}^{|\mathcal{D}|}} \Big(& \overbrace{\log p_{\boldsymbol{\lambda}}(y_i \mid \textbf{x}_i)}^{\textrm{log likelihood}}  + \\
& \vspace{-1cm} \tau \underbrace{\mathbb{E}_{p_{\boldsymbol{\lambda}}(y' \mid \textbf{x}_i)} \log p_{\boldsymbol{\lambda}}(y' \mid \textbf{x}_i)}_{-\textrm{conditional entropy}}\Big)
\end{align*}
}

%%%%%%%%%%%%%%%%%%%%%%%%%%%%%%%%%%%%%%%%%Main Features%%%%%%%%%%%%%%%%%%%%%%%%%%%%%%%%%%%%%%
\headerbox{Features}{name=features,column=0,span=1.15,below=model}{
%\linebreak 

\begin{description}
\item [Part-of-speech (POS) $n$-grams] Counts of every POS 1-, 2-, and 3-gram in each document. %53.92\% 
\item [FreqChar] Counts of character 1--4-grams that are observed more than 5 times in the training corpus.
\item [CharPrompt] Conjunction of the FreqChar features with the prompt ID %65.09
\item [Brown clusters] We clustered 8~billion words of English into 600~clusters and used 1--4-grams.
\item[PsvRatio] The proportion of passive verbs out of all verbs.
\item[DocLen] Document length in tokens.
\item[Punct] Counts of each punctuation mark. 
\item[Misspell] Spelling correction edits. Features included substitutions, deletions, insertions and joinings.% as well as the word position where the error occurred.
\item[\feat{Restore}] Substitutions, deletions and
  insertions of common words that were restored with an $n$-gram LM.
\item[CxtFxn] Contextual function words. Counts of $n$-grams consisting of one or two function words and the POS tag of the adjacent words: {\tt CHI:<some JJ>}.

\end{description}
}  
%%%%%%%%%%%%%%%%%%%%%%%%%%%%%%%%%%%%%%%%%Example%%%%%%%%%%%%%%%%%%%%%%%%%%%%%%%%%%%%%%
\headerbox{Example: L1 German sentence}{name=example,column=1.15,span = 1.85}{ 
%We focus on two broad novel categories of features: those inspired by the features used to identify translationese and those extracted by automatic statistical ``correction'' of the essays.

\centering\small
\vspace{4pt}
\textbf{Firstly the employers live more savely because they are going to have more money to spend for luxury .}\\[14pt]

\begin{tabular}{p{6em}p{15em}p{15em}}
 & \multicolumn{1}{c}{\bf Presence} & \multicolumn{1}{c}{\bf Considered alternatives/edits} \\
\hline
\bf Characters
& 
\smaller[.5]
 \begin{tabular}{@{}p{12em}p{5em}@{}} 
   "FreqChar\_l\_y\_ ":&$\log{2+1}$\\"CharPrompt\_P5\_g\_o\_i":&$\log{1+1}$\\"Punct\_period":&$\log{1+1}$ %\\"MFChar\_e\_ ":&$\log{1+1}$
 \end{tabular} 
& 
\smaller[.5]
 \begin{tabular}{@{}p{12em}p{5em}@{}} 
   "Misspell\_DeleteP\_p\_.":&$1.0$\\"Misspell\_InsertP\_p\_,":& $1.0$\\"Misspell\_MID:SUBST:v:f":&$\log{1+1}$\\"Misspell\_SUBST:v:f":&$\log{1+1}$
 \end{tabular} \\
\hline
\bf Words 
& 
\smaller[.5]
 \begin{tabular}{@{}p{12em}p{5em}@{}}
    "DocLen\_":&$\log{19+1}$\\"MeanWordRank":&$422.6$\\"CohMarker\_because":&$\log{1+1}$\\"MostFreq\_have":&$\log{1+1}$\\"PosToken\_last\_luxury":&$\log{1+1}$\\ "Pronouns\_they":&$\log{1+1}$
 \end{tabular}
& 
\smaller[.5]
 \begin{tabular}{@{}p{12em}p{5em}@{}}
   "Misspell\_safely":&$\log{1+1}$\\"Restore\_Match\_p\_to":&$0.5$\\"Restore\_Delete\_p\_to":&$0.5$\\"Restore\_Delete\_p\_are":&$1.0$\\"Restore\_Delete\_p\_because":&$1.0$\\"Restore\_Delete\_p\_for":&$1.0$
 \end{tabular} \\
\hline
\bf{POS} 
& 
\smaller[.5]
\begin{tabular}{@{}p{12em}p{5em}@{}}
  "POS\_VBP\_VBG\_TO":&$\log{1+1}$\\"POS\_p\_VBP\_VBG\_TO":&$0.059$
\end{tabular} \\
\hline
\bf Words + POS 
& 
\smaller[.5]
\begin{tabular}{@{}p{12em}p{5em}@{}}
  "CxtFxn\_VBP\_VBG\_to":&$\log{1+1}$\\"CxtFxn\_more\_RB":&$\log{1+1}$
\end{tabular} \\
\hline

\bf Brown
& 
\smaller[.5]
\begin{tabular}{@{}p{30.5em}p{5em}@{}}
  "C\_1111101111110\_110100011110\_110101101100":
  & $\log{1+1}$\\
  
\end{tabular} \\\\[14pt]

\end{tabular}

\begin{tabular}{p{10em}p{32em}}
  \multicolumn{1}{c}{\bf Brown clusters} & \multicolumn{1}{c}{\bf Words in cluster}\\\hline
 C\_1111101111110   & investors customers patients employees consumers users citizens shareholders clients individuals managers buyers viewers \textbf{employers} guests readers immigrants taxpayers humans donors households homeowners competitors travelers audiences borrowers shoppers offenders physicians creditors subscribers stockholders sellers entrepreneurs advertisers applicants motorists tenants builders smokers strangers collectors listeners savers retirees outsiders travellers bidders bondholders patrons\\ \hline
  
 C\_110100011110   & \textbf{live} remain stay stand die sit compete operate invest participate arrive engage succeed lie cope gather testify comply communicate proceed weigh disagree cooperate intervene expire rein behave interact thrive interfere prevail persist coincide explode collaborate linger grips enroll indulge resonate dine tread prosper loom grapple reside retaliate collide regroup innovate \\ \hline
 
  C\_110101101100   & \textbf{more} less fewer ... \\\\
 
 \end{tabular} 
% \begin{center}
%Some of the features extracted for an L1 German sentence.
%\end{center}
}

%%%%%%%%%%%%%%%%%%%%%%%%%%%%%%%%%%%%%%%%%Results%%%%%%%%%%%%%%%%%%%%%%%%%%%%%%%%%%%%%%
\headerbox{Results}{name=results,column=1.15,span=1.85,below=example}{
\begin{center}
\small
\begin{tabular}{|l|r|r|r|r|r|r|r|r|r|r|r||c|c|}
      \hline
 \multirow{2}{*}{} &ARA & CHI & FRE & GER & HIN & ITA & JPN &  KOR & SPA &  TEL &  TUR	& \bf P(\%) & \bf R(\%)  \\			
      \hline\hline
ARA & 80 & 0 & 2 & 1 & 3 & 4 & 1 & 0 & 4 & 2 & 3 & 80.8 & 80.0  \\
CHI & 3 & 80 & 0 & 1 & 1 & 0 & 6 & 7 & 1 & 0 & 1 & 88.9 & 80.0  \\
\hline
FRE & 2 & 2 & 81 & 5 & 1 & 2 & 1 & 0 & 3 & 0 & 3 & 86.2 & 81.0  \\
GER & 1 & 1 & 1 & 93 & 0 & 0 & 0 & 1 & 1 & 0 & 2 & 87.7 & 93.0  \\
\hline
HIN & 2 & 0 & 0 & 1 & 77 & 1 & 0 & 1 & 5 & 9 & 4 & 74.8 & 77.0  \\
ITA & 2 & 0 & 3 & 1 & 1 & 87 & 1 & 0 & 3 & 0 & 2 & 82.1 & 87.0   \\
\hline
JPN & 2 & 1 & 1 & 2 & 0 & 1 & 87 & 5 & 0 & 0 & 1 & 78.4 & 87.0 \\
KOR & 1 & 5 & 2 & 0 & 1 & 0 & 9 & 81 & 1 & 0 & 0 & 80.2 & 81.0  \\
\hline
SPA & 2 & 0 & 2 & 0 & 1 & 8 & 2 & 1 & 78 & 1 & 5 & 77.2 & 78.0  \\
TEL & 0 & 1 & 0 & 0 & 18 & 1 & 2 & 1 & 1 & 73 & 3 & 85.9 & 73.0   \\
%\hline
TUR & 4 & 0 & 2 & 2 & 0 & 2 & 2 & 4 & 4 & 0 & 80 & 76.9 & 80.0   \\
			\hline%\hline
\end{tabular}
\end{center}
\begin{center}
Official test set confusion matrix with the full model. Accuracy on the test set is 81.5\%\\
\end{center}

}

%%Additional features:
%\begin{description}
%\item[DocLen] Document length in tokens. 11.81\%.
%\item[Punct] Counts of each punctuation mark. 27.41\%.
%\item[Pron] Counts of each pronoun. 22.81\%.
%\item[Position] Positional token frequency. We use the counts for the first two and last three words before the period in each sentence as features. 53.03\%.
%\item[PsvRatio] The proportion of passive verbs out of all verbs. 12.26\%.
%\item[CxtFxn] Contextual function words. Bi-grams yield 62.79\%, tri-gram 62.32\%.
%\item[Misspell] Spelling correction edits. ???. 37.29\%.
%\item[Restore] Counts of substitutions, deletions and insertions of predefined tokens that we restored in the texts. 47.67\
%\end{description}
%
%\begin{center}
%\begin{tabular}{lrc}
%\small
%\textbf{Feature Group} & \multicolumn{1}{c}{\# Params}  & \multicolumn{1}{c}{Accuracy(\%)} \\\hline
%%\feat{POS} & 540,947 & 55.18 \\
%\textsc{main} & 5,664,461 & 81.09 \\
%\textsc{main} + \feat{Position} & 6,153,015 & 81.00 \\
%\textsc{main} + \feat{PsvRatio} & 5,664,472 & 81.00 \\
%\textsc{main} + \feat{DocLen} & 5,664,472 & 81.09  \\
%\textsc{main} + \feat{Pron} & 5,664,736 & 81.09  \\
%\textsc{main} + \feat{Punct} & 5,664,604 & 81.09  \\
%\textsc{main} + \feat{Misspell} & 5,799,860 & 81.27  \\
%\textsc{main} + \feat{Restore} & 5,682,589 & 81.36  \\
%\textsc{main} + \feat{CxtFxn} & 7,669,684 & 81.73 \\
%%+ \feat{FreqChar} & 1,036,871 & 79.55 \\ 
%%\quad + \feat{CharPrompt} & 2,111,175 & 79.82 \\ 
%%\qquad + \feat{Brown} & 5,664,461 & 81.09 \\
%\end{tabular}
%\end{center}
%}

%%%%%%%%%%%%%%%%%%%%%%%%%%%%%%%%%%%%%%%%%Results%%%%%%%%%%%%%%%%%%%%%%%%%%%%%%%%%%%%%% 
  \headerbox{Accuracy}{name=results,column=3,span=1}{
The full model that we used to classify the test set combines all features. %The accuracy of the full model on the test set is 81.5\%. %listed in \tref{tbl:addfeats}. 
%In this model, the accuracy  for 10-fold cross-validation on the development set is {\bf ~84.55\% }, and on the test set it is~81.5\%. %most of the additional features exhibited negligible improvement.

\begin{center}
\begin{tabular}{|l|c|c|c|}
      \hline
 \multirow{2}{*} {} Main features & \# Params 	& Accuracy(\%)	\\				
       \hline\hline
\feat{POS} & 540,947 & 55.18 \\
+ \feat{FreqChar} & 1,036,871 & 79.55 \\ 
\quad + \feat{CharPrompt} & 2,111,175 & 79.82 \\ 
\qquad + \feat{Brown} & 5,664,461 & 81.09 \\
			\hline%\hline
\end{tabular}
\end{center}

\begin{center}
\begin{tabular}{|l|c|c|c|}
      \hline
 \multirow{2}{*} {} Additional features & \# Params 	& Accuracy(\%)	\\				
       \hline\hline
\textsc{main} & 5,664,461 & 81.09 \\
%\textsc{main} + \feat{Position} & 6,153,015 & 81.00 \\
\textsc{main} + \feat{PsvRatio} & 5,664,472 & 81.00 \\
\textsc{main} + \feat{DocLen} & 5,664,472 & 81.09  \\
%\textsc{main} + \feat{Pron} & 5,664,736 & 81.09  \\
\textsc{main} + \feat{Punct} & 5,664,604 & 81.09  \\
\textsc{main} + \feat{Misspell} & 5,799,860 & 81.27  \\
\textsc{main} + \feat{Restore} & 5,682,589 & 81.36  \\
\textsc{main} + \feat{CxtFxn} & 7,669,684 & 81.73 \\
\textsc{full model} & - & {\bf 84.55} \\

			\hline%\hline
\end{tabular}
\end{center}
\begin{center}
10-fold cross-validation on the development set.
\end{center}
}

 
  
%%%%%%%%%%%%%%%%%%%%%%%%%%%%%%%%%%%%%%Results Analysis%%%%%%%%%%%%%%%%%%%%%%%%%%%%%%%%%%%%%%%%
  \headerbox{Analysis}{name=analysis,column=3,span=1,below=results}{
 % \scriptsize 
Texts produced by non native English writers involve a tension between the imposing models of the native language, on the one hand, and a set of cognitive constraints resulting from the efforts to generate the target text, on the other. 
The former is called \emph{interference} in Translation Studies. %\cite{Toury:1995}. 
%\linebreak
We explore the effects of interference by analyzing several patterns we observe in the features. 

\begin{itemize}
\item Arabic speakers use \textnl{a lot} as a single word more often and sometime omit the definite article before nouns and adjectives. 
\item German authors use hyphens more frequently, probably due to compounding in their native language. They also tend to substitute the letter \textnl{y} with \textnl{z} and vice versa.
%German authors use Hyphens more frequently and tending to substitute the letter \textnl{y} with \textnl{z} and vice versa. We suspect this owes to their switched positions on German keyboards. 
\item Japanese authors confuse \textnl{l} and \textnl{r}.
\item The characters \textnl{r} and \textnl{s} are misused in Chinese
and Spanish, respectively.
%\item The word \textnl{then} is dominant in the texts of Hindi speakers.
\end{itemize}

%\begin{itemize}
%\item Character sequence \textnl{alot} is overrepresented in English written by Arabic speakers.  %and we speculate that Arabic speakers conceive \textnl{a lot} as a single word; For the same reason, another prominent feature is 
%\item A missing definite article before nouns and adjectives. Arabic has no indefinite article.
%\item Essays that contain hyphens are more likely to be from German authors. 
%\item Essays by native Germans is a frequent substitution of the letter \textnl{y} for \textnl{z} and vice versa. We suspect this owes to their switched positions on German keyboards.
%%Another unexpected feature of essays by native Germans is a frequent substitution of the letter \textnl{y} for \textnl{z} and vice versa. We suspect this owes to their switched positions on German keyboards. 
%\item Frequent misspellings involving confusions of \textnl{l} and \textnl{r} in Japanese essays.
%\end{itemize}
}



%\begin{center}
%\begin{tabular}{|l|r|r|r|r|r|r|r|r|r|r|r|}
%\hline
%%\small
%\multirow{2}{*}{} & ARA & CHI & FRE & GER & HIN & ITA & JPN &  KOR & SPA &  TEL &  TUR & \\				
%\hline\hline
%ARA & 80 & 0 & 2 & 1 & 3 & 4 & 1 & 0 & 4 & 2 & 3 & 80.8 & 80.0 & 80.4 \\
%CHI & 3 & 80 & 0 & 1 & 1 & 0 & 6 & 7 & 1 & 0 & 1 & 88.9 & 80.0 & 84.2 \\
%FRE & 2 & 2 & 81 & 5 & 1 & 2 & 1 & 0 & 3 & 0 & 3 & 86.2 & 81.0 & 83.5 \\
%%\hline
%GER & 1 & 1 & 1 & 93 & 0 & 0 & 0 & 1 & 1 & 0 & 2 & 87.7 & 93.0 & 90.3 \\
%HIN & 2 & 0 & 0 & 1 & 77 & 1 & 0 & 1 & 5 & 9 & 4 & 74.8 & 77.0 & 75.9 \\
%ITA & 2 & 0 & 3 & 1 & 1 & 87 & 1 & 0 & 3 & 0 & 2 & 82.1 & 87.0 & 84.5 \\
%%\hline
%JPN & 2 & 1 & 1 & 2 & 0 & 1 & 87 & 5 & 0 & 0 & 1 & 78.4 & 87.0 & 82.5 \\
%KOR & 1 & 5 & 2 & 0 & 1 & 0 & 9 & 81 & 1 & 0 & 0 & 80.2 & 81.0 & 80.6 \\
%SPA & 2 & 0 & 2 & 0 & 1 & 8 & 2 & 1 & 78 & 1 & 5 & 77.2 & 78.0 & 77.6 \\
%%\hline
%TEL & 0 & 1 & 0 & 0 & 18 & 1 & 2 & 1 & 1 & 73 & 3 & 85.9 & 73.0 & 78.9 \\
%TUR & 4 & 0 & 2 & 2 & 0 & 2 & 2 & 4 & 4 & 0 & 80 & 76.9 & 80.0 & 78.4 \\
%\hline\hline
%\end{tabular}
%\end{center}
%\begin{center}
%C
%\end{center}


%\begin{center}
%%\begin{tabular}{>{\bf}l|r@{ }r@{ }r@{ }r@{ }r@{ }r@{ }r@{ }r@{ }r@{ }r@{ }r|ccc}
%\begin{tabular}{l|rrrrrrrrrrr|ccc}
%\small
%& \bf ARA & \bf CHI & \bf FRE & \bf GER & \bf HIN & \bf ITA & \bf JPN & \bf KOR & \bf SPA & \bf TEL & \bf TUR & \bf Precision (\%) & \bf Recall (\%) & \bf $F_1$ (\%) \\
%\hline
%ARA & 80 & 0 & 2 & 1 & 3 & 4 & 1 & 0 & 4 & 2 & 3 & 80.8 & 80.0 & 80.4 \\
%CHI & 3 & 80 & 0 & 1 & 1 & 0 & 6 & 7 & 1 & 0 & 1 & 88.9 & 80.0 & 84.2 \\
%FRE & 2 & 2 & 81 & 5 & 1 & 2 & 1 & 0 & 3 & 0 & 3 & 86.2 & 81.0 & 83.5 \\
%\hline
%GER & 1 & 1 & 1 & 93 & 0 & 0 & 0 & 1 & 1 & 0 & 2 & 87.7 & 93.0 & 90.3 \\
%HIN & 2 & 0 & 0 & 1 & 77 & 1 & 0 & 1 & 5 & 9 & 4 & 74.8 & 77.0 & 75.9 \\
%ITA & 2 & 0 & 3 & 1 & 1 & 87 & 1 & 0 & 3 & 0 & 2 & 82.1 & 87.0 & 84.5 \\
%\hline
%JPN & 2 & 1 & 1 & 2 & 0 & 1 & 87 & 5 & 0 & 0 & 1 & 78.4 & 87.0 & 82.5 \\
%KOR & 1 & 5 & 2 & 0 & 1 & 0 & 9 & 81 & 1 & 0 & 0 & 80.2 & 81.0 & 80.6 \\
%SPA & 2 & 0 & 2 & 0 & 1 & 8 & 2 & 1 & 78 & 1 & 5 & 77.2 & 78.0 & 77.6 \\
%\hline
%TEL & 0 & 1 & 0 & 0 & 18 & 1 & 2 & 1 & 1 & 73 & 3 & 85.9 & 73.0 & 78.9 \\
%TUR & 4 & 0 & 2 & 2 & 0 & 2 & 2 & 4 & 4 & 0 & 80 & 76.9 & 80.0 & 78.4 \\
%\end{tabular}
%\end{center}


%\begin{center}
% \includegraphics[width=0.67\textwidth]{img/graph}
%\end{center}


%%%%%%%%%%%%%%%%%%%%%%%%%%%%%%%%%%%%%%%%References%%%%%%%%%%%%%%%%%%%%%%%%%%%%%%%%%%%%%%%%
%\headerbox{References}{name=references,column=3,span=1,below=analysis}{
%    \smaller
%    \vspace{-1.2em}
%    \renewcommand{\refname}{}
%\bibliographystyle{plainnat}
%\bibliography{l1id}
%  }

%%%%%%%%%%%%%%%%%%%%%%%%%%%%%%%%%%%%%%%%Acknowledgements%%%%%%%%%%%%%%%%%%%%%%%%%%%%%%%%%%%%%%
\headerbox{Acknowledgements}{name=ack,column=3,span=1,below=analysis}{
%\scriptsize 
    This research was supported by a grant from the Israeli Ministry of Science and Technology.  
}

\end{poster}%
\end{document}
